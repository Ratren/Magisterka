\documentclass[a4paper,12pt]{book} % nie: report!


\usepackage[T1,plmath]{polski} % lepiej to zamiast babel!
\usepackage[utf8]{inputenc} % w razie kłopotów spróbować: \usepackage[utf8x]{inputenc}
\usepackage{fancyhdr} % nagłówki i stopki
\usepackage{indentfirst} % WAŻNE, MA BYĆ!
\usepackage[pdftex]{graphicx} % to do wstawiania rysunków
\usepackage{amsfonts} % pakiety od AMS, ułatwiają składanie pewnych techniczno-matematcyznych rzeczy
\usepackage{amsmath} % to do dodatkowych symboli, przydatne
\usepackage{amssymb} % to też do dodatkowych symboli, też przydatne
\usepackage{amsthm}
\usepackage[pdftex,
            left=1.1in,right=1.1in,
            top=1.1in,bottom=1.1in]{geometry} % marginesy
\usepackage{float}
\usepackage[font=small,labelfont=bf]{caption}
\usepackage[backend=biber, style=numeric, citestyle=numeric, language=polish, sortlocale=pl_PL]{biblatex}
\addbibresource{bibliografia.bib}


\usepackage[colorlinks=true]{hyperref} % odnośniki interaktywne w PDFie
\hypersetup{allcolors=blue}
\usepackage{listings}
\lstset{
    basicstyle=\footnotesize\tt,
    numbers=left,
    numberstyle=\tiny,
    frame=tb,
    tabsize=4,
    columns=fixed,
    showstringspaces=false,
    showtabs=false,
    keepspaces,
    commentstyle=\color{red},
    keywordstyle=\color{blue}
}
\newfloat{lstfloat}{htbp}{lolst}[chapter]
\floatname{lstfloat}{Listing}
\def\lstfloatautorefname{Listing}

% jesli potrzeb, można oczywiście wstawić inne pakiety i swoje definicje...



% definicje nagłówków i stopek
\pagestyle{fancy}
\renewcommand{\chaptermark}[1]{\markboth{#1}{}}
\renewcommand{\sectionmark}[1]{\markright{\thesection\ #1}}
\fancyhf{}
\fancyhead[LE,RO]{\footnotesize\bfseries\thepage}
\fancyhead[LO]{\footnotesize\rightmark}
\fancyhead[RE]{\footnotesize\leftmark}
\renewcommand{\headrulewidth}{0.5pt}
\renewcommand{\footrulewidth}{0pt}
\addtolength{\headheight}{1.5pt}
\fancypagestyle{plain}{\fancyhead{}\cfoot{\footnotesize\bfseries\thepage}\renewcommand{\headrulewidth}{0pt}}


% interlinia
\linespread{1.25}



\begin{document}
\begin{titlepage}
% ~

\begin{tabular}{c@{\hspace{21mm}}|@{\hspace{5mm}}l}
\vspace{-20mm} & \\
\multicolumn{2}{l}{\hspace{-12.5mm} \includegraphics[width=8cm]{LogoUMCS.jpg}} \\
\multicolumn{2}{@{\hspace{20mm}}l}{\vspace{-4mm}} \\
\multicolumn{2}{@{\hspace{28mm}}l}{\Large \sf UNIWERSYTET MARII
	CURIE-SKŁODOWSKIEJ} \\
\multicolumn{2}{@{\hspace{28mm}}l}{\vspace{-4mm}} \\
\multicolumn{2}{@{\hspace{28mm}}l}{\Large \sf W LUBLINIE} \\
\multicolumn{2}{@{\hspace{28mm}}l}{\vspace{-4mm}} \\
\multicolumn{2}{@{\hspace{28mm}}l}{\Large \sf Wydział Matematyki, Fizyki i
	Informatyki} \\
\multicolumn{2}{@{\hspace{28mm}}l}{\vspace{21mm}} \\
& {\sf Kierunek: \textbf{informatyka} } \\
& \\\\\\
& {\sf \large \bfseries Rafał Lenart} \\
& {\sf nr albumu: 307726} \\
& \\\\\\
& \Large \sf \bfseries Pamięć podręczna jako narzędzie\\
& \Large \sf \bfseries optymalizacji procesów obliczeniowych\\
& \Large \sf \bfseries (jakoś tak ale może nie) \\\\[-10pt]
& {\large \sf Cache memory as a tool for } \\
& {\large \sf optimizing computational processes} \\
& \\
& \\
& \\
& {\sf Praca magisterska}  \\
& \vspace{-7mm} \\
&  {\sf napisana w Katedrze Cyberbezpieczeńtwa i lingwistyki komputerowej} \\
&  {\sf Instytutu Informatyki UMCS} \\
& \vspace{-7mm} \\
& {\sf pod kierunkiem \bfseries dr hab. Jarosława Byliny lub dr hab. Beaty Byliny} \\
\multicolumn{2}{@{\hspace{28mm}}l}{\vspace{15mm}} \\
\multicolumn{2}{@{\hspace{28mm}}l}{\textbf{\textsf{Lublin 2022}}}
\end{tabular}
\end{titlepage}





\sloppy



\thispagestyle{empty}


\newpage{}

\thispagestyle{empty}

\newpage{}



\tableofcontents{}

\chapter*{Wstęp}
\addcontentsline{toc}{chapter}{Wstęp} 
\chaptermark{Wstęp}
\cite{demaine2015cache}\cite{drepper2007every}\cite{przybylski1990cache}\cite{ademodi2020cache}

\chapter{Budowa i hierarchia pamięci komputera}
\section{DRAM}
\section{SRAM}
\section{Ogólna hierarchia pamięci}
\section{Poziomy pamięci podręcznej}
\section{Adresowanie}

\chapter{Optymalizacje użycia pamięci podręcznej}
\section{Wymienić kilka typu prefetching itp. oraz je opisać}
\section{Oczywiście tylko te użyte w pracy}
\section{Cache-aware vs. Cache-oblivious}

\chapter{BLAS}
\section{Czym jest BLAS oraz jego poziomy}
\section{Dlaczego ważny w pracy}
\section{Własna implementacja niektórych funkcji oraz jej opis.}

\chapter{Algorytm}
\section{LCS, algorytm Nussinova, FFT}
\section{nie wiem co wybrać}
\section{Opis algorytmu}
\section{Znane implementacje}
\section{Modyfikacja bazowego algorytmu z optymalizacjami}

\chapter{Opis implementacji}
\section{Przegląd kodu}
\section{Omówienie wyników testów}
\section{Pokazane przyspieszenie (porównane ze znaną implementacją?)}

\chapter{Czegoś tu brakuje ale nie mam pojęcia co dodać.}
\section{Do wybrania jest algorytm}
\section{Czy coś jeszcze jest potrzebne w pracy?}


\chapter*{Podsumowanie} % jak we wstępie
\addcontentsline{toc}{chapter}{Podsumowanie}
\chaptermark{Podsumowanie}


\listof{lstfloat}{Spis listingów} % jeśli są listingi
\addcontentsline{toc}{chapter}{Spis listingów}

\listoftables{} % jeśli są tabele
\addcontentsline{toc}{chapter}{Spis tabel}

\listoffigures{} % jeśli są rysunki
\addcontentsline{toc}{chapter}{Spis rysunków}

\printbibliography


\end{document}
